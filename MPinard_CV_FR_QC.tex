\documentclass[11pt,a4paper]{moderncv}

%----------------------------------------------------------------------------------------
%   PACKAGES
%----------------------------------------------------------------------------------------

\usepackage[utf8]{inputenc}
%\usepackage[scale=0.975]{geometry}
\usepackage[top=0.5cm, bottom=0.5cm, left=0.5cm, right=0.5cm]{geometry}
\usepackage{graphicx}
\usepackage[firstyear=2011,lastyear=2018]{moderntimeline}
\usepackage{datenumber,fp}
\usepackage{fontawesome}
%\usepackage{academicons}

%----------------------------------------------------------------------------------------
%   CONFIG
%----------------------------------------------------------------------------------------

%-------------------------------------------
%   Packages internal commands redefinitions
%-------------------------------------------

%--> adding preinfo

%\makeatletter % changes the catcode of @ to 11
%modify here packages internal commands
%\makeatother % changes the catcode of @ back to 12

\makeatletter
\newcommand*{\preinfo}[1]{\def\@preinfo{#1}}

% optional maketitle width to force a certain width (if set to 0pt, the width is calculated automatically)
\newlength{\makecvtitlenamewidth}
\setlength{\makecvtitlenamewidth}{0pt}% dummy value
\renewcommand*{\makecvtitle}{%
  % recompute lengths (in case we are switching from letter to resume, or vice versa)
  \recomputecvlengths%
  % optional detailed information (pre-rendering)
  \def\phonesdetails{}%
  \collectionloop{phones}{% the key holds the phone type (=symbol command prefix), the item holds the number
    \protected@edef\phonesdetails{\phonesdetails\protect\makenewline\csname\collectionloopkey phonesymbol\endcsname\collectionloopitem}}%
  \def\socialsdetails{}%
  \collectionloop{socials}{% the key holds the social type (=symbol command prefix), the item holds the link
    \protected@edef\socialsdetails{\socialsdetails\protect\makenewline\csname\collectionloopkey socialsymbol\endcsname\collectionloopitem}}%
  \newbox{\makecvtitledetailsbox}%
  \savebox{\makecvtitledetailsbox}{%
    \addressfont\color{color2}%
    \begin{tabular}[b]{@{}r@{}}%
    \ifthenelse{\isundefined{\@preinfo}}{}{\makenewline{\@preinfo}}%
      \ifthenelse{\isundefined{\@addressstreet}}{}{\makenewline\addresssymbol\@addressstreet%
        \ifthenelse{\equal{\@addresscity}{}}{}{\makenewline\@addresscity}% if \addresstreet is defined, \addresscity and addresscountry will always be defined but could be empty
        \ifthenelse{\equal{\@addresscountry}{}}{}{\makenewline\@addresscountry}}%
      \phonesdetails% needs to be pre-rendered as loops and tabulars seem to conflict
      \ifthenelse{\isundefined{\@email}}{}{\makenewline\emailsymbol\emaillink{\@email}}%
      \ifthenelse{\isundefined{\@homepage}}{}{\makenewline\homepagesymbol\httplink{\@homepage}}%
      \socialsdetails% needs to be pre-rendered as loops and tabulars seem to conflict
      \ifthenelse{\isundefined{\@extrainfo}}{}{\makenewline\@extrainfo}%
    \end{tabular}
  }%
  % optional photo (pre-rendering)
  \newbox{\makecvtitlepicturebox}%
  \savebox{\makecvtitlepicturebox}{%
    \ifthenelse{\isundefined{\@photo}}%
    {}%
    {%
      \hspace*{\separatorcolumnwidth}%
      \color{color1}%
      \setlength{\fboxrule}{\@photoframewidth}%
      \ifdim\@photoframewidth=0pt%
        \setlength{\fboxsep}{0pt}\fi%
      \framebox{\includegraphics[width=\@photowidth]{\@photo}}}}%
  % name and title
  \newlength{\makecvtitledetailswidth}\settowidth{\makecvtitledetailswidth}{\usebox{\makecvtitledetailsbox}}%
  \newlength{\makecvtitlepicturewidth}\settowidth{\makecvtitlepicturewidth}{\usebox{\makecvtitlepicturebox}}%
  \ifthenelse{\lengthtest{\makecvtitlenamewidth=0pt}}% check for dummy value (equivalent to \ifdim\makecvtitlenamewidth=0pt)
    {\setlength{\makecvtitlenamewidth}{\textwidth-\makecvtitledetailswidth-\makecvtitlepicturewidth}}%
    {}%
  \begin{minipage}[b]{\makecvtitlenamewidth}%
    \namestyle{\@firstname\ \@lastname}%
    \ifthenelse{\equal{\@title}{}}{}{\raggedright\\[1.25em]\titlestyle{\@title}}
  \end{minipage}%
  \hfill%
  % optional detailed information (rendering)
  \llap{\usebox{\makecvtitledetailsbox}}% \llap is used to suppress the width of the box, allowing overlap if the value of makecvtitlenamewidth is forced
  % optional photo (rendering)
  \usebox{\makecvtitlepicturebox}\\[2.5em]%
  % optional quote
  \ifthenelse{\isundefined{\@quote}}%
    {}%
    {{\centering\begin{minipage}{\quotewidth}\centering\quotestyle{\@quote}\end{minipage}\\[2.5em]}}%
  \par}% to avoid weird spacing bug at the first section if no blank line is left after \makecvtitle
\makeatother

%-------------------------------------------
%   Theme related colors
%-------------------------------------------

% color0      main default color, normally left to black
% color1      primary scheme color
% color2      secondary scheme color

%-------------------------------------------
%   Theme font and style redefinition
%-------------------------------------------
\AtBeginDocument{%

% fonts
\renewcommand*{\namefont}{\fontsize{38}{45}\mdseries\upshape}
\renewcommand*{\titlefont}{\fontsize{13}{14}\mdseries\upshape}
%\renewcommand*{\addressfont}{\normalsize\mdseries\upshape}
%\renewcommand*{\quotefont}{\large\slshape}
%\renewcommand*{\sectionfont}{\Large\mdseries\upshape}
%\renewcommand*{\subsectionfont}{\large\mdseries\upshape}
%\renewcommand*{\hintfont}{}

% styles
%\renewcommand*{\namestyle}[1]{{\namefont\textcolor{color1}{#1}}}
%\renewcommand*{\titlestyle}[1]{{\titlefont\textcolor{color2!85}{#1}}}
%\renewcommand*{\addressstyle}[1]{{\addressfont\textcolor{color2}{#1}}}
%\renewcommand*{\quotestyle}[1]{{\quotefont\textcolor{color1}{#1}}}
%\renewcommand*{\sectionstyle}[1]{{\sectionfont\textcolor{color1}{#1}}}
%\renewcommand*{\subsectionstyle}[1]{{\subsectionfont\textcolor{color1}{#1}}}
%\renewcommand*{\hintstyle}[1]{{\hintfont\textcolor{color0}{#1}}}

}

%-------------------------------------------
%   Symbols from FontAwesome
%-------------------------------------------

\AtBeginDocument{%

\renewcommand*{\addresssymbol}       {\faMapMarker\ }
\renewcommand*{\mobilephonesymbol}   {\faMobile\ }
\renewcommand*{\fixedphonesymbol}    {\foPhone\ }
\renewcommand*{\faxphonesymbol}      {\faFax\ }
\renewcommand*{\emailsymbol}         {\faEnvelopeO\ }
\renewcommand*{\homepagesymbol}      {\faGlobe\ }
\renewcommand*{\linkedinsocialsymbol}{\faLinkedin\ }
\renewcommand*{\twittersocialsymbol} {\faTwitter\ }
\renewcommand*{\githubsocialsymbol}  {\faGithub\ }

}

%-------------------------------------------
%   Commands definitions
%-------------------------------------------

% Used to indicate the level in the languages section
\newcommand{\cvskill}[2]{%
\textcolor{color2}{\textbf{#1}}
\foreach \x in {1,...,5}{%
  \space{\ifnumgreater{\x}{#2}{\color{color1!30}}{\color{color1}}\faCircle}}%
}

% Space to delete between the sections
\newcommand{\deletedSpace}{-2.95mm}

%-------------------------------------------
%   Age computation
%-------------------------------------------

\newcounter{birthday}
\newcounter{today}
\setmydatenumber{birthday}{1996}{06}{19}
\setmydatenumber{today}{\the\year}{\the\month}{\the\day}
\FPsub\result{\thetoday}{\thebirthday}
\FPdiv\myage{\result}{365.2425}
\FPround\myage{\myage}{0}

%-------------------------------------------
%   ModernCV theme
%-------------------------------------------

\moderncvtheme[blue]{banking}
%\moderncvtheme[blue]{classic}
%\renewcommand{\familydefault}{\sfdefault}
\nopagenumbers{}

%----------------------------------------------------------------------------------------
%   PERSONAL DATA
%----------------------------------------------------------------------------------------

\preinfo{%
  \faUser\ \myage{} ans, célibataire
}
\firstname{Maxime}
\familyname{Pinard}
\title{Élève-ingénieur informatique en 1ème année (\textit{BAC+3}), je cherche un stage de développement pour automne 2017}
\address{2 Avenue d'Alsace}{70400 Héricourt, France}
\mobile{+33~(0)~687~925~509}
%\phone{}
%\fax{}
\email{maxime.pin@live.fr}
%\homepage{maxime.pinard.info}
\extrainfo{%
  \linkedinsocialsymbol\ \httplink{linkedin.com/in/maxime-pinard}\\
  \githubsocialsymbol\ \httplink{github.com/pinam45}
}
\photo[70pt]{MPinard.jpg}
%\quote{Objectif: Intégrer une équipe au sein d'un projet novateur en tant que développeur informatique.}
%\quote{Passionné d'informatique et de jeux vidéo, je me forme pour travailler sur des projets a la rencontre de mes deux passions.}


%----------------------------------------------------------------------------------------
%   DOCUMENT BODY
%----------------------------------------------------------------------------------------

\begin{document}
	\maketitle
	\vspace*{-16mm}

%-------------------------------------------
%   Course
%-------------------------------------------

	\section{Cursus}
		\tlcventry{2016}{0}{Diplôme d'ingénieur informatique, spécialité imagerie}{Université de Technologie de Belfort-Montbéliard}{Belfort, France}{\textit{Bac+5 (eq. Master degree)}}{Actuellement en 3ème année, \textit{Bac+3 (eq. Bachelor degree)}}
		\tlcventry{2014}{2016}{DEUTEC}{Université de Technologie de Belfort-Montbéliard}{Sèvenans, France}{\textit{Bac+2}}{Tronc Commun avant choix de la filière diplômante}
		\tlcventry{2011}{2014}{Baccalauréat S, option SVT, spécialité Mathématiques}{Lycée Louis Aragon}{Héricourt, France}{\small\textit{Mention Bien (eq. DEC)}}{}

%-------------------------------------------
%   Others qualifications
%-------------------------------------------

  \vspace*{\deletedSpace}
	\section{Certifications autres}
		\cvlistitem{Permis de conduire}{}
		\cvlistitem{Prévention et Secours Civiques de niveau 1 (PSC1)}{}

%-------------------------------------------
%   Computer skills
%-------------------------------------------

  \vspace*{\deletedSpace}
	\section{Compétences informatiques}
		\subsection{Langages}
			%\cvdoubleitem{Maitrisés}{C, C++, JAVA, \LaTeX ,HTML/CSS.}{Connus}{VBA, UPC, CUDA, PHP, SQL, VHDL, Assembleur.}
      \cvline{Usage fréquent}{C, C++, Java (SE et EE), \LaTeX ,HTML/CSS.}
			\cvline{Connus}{C\#, VBA, UPC, CUDA, PHP, SQL, VHDL, Assembleur.}
		\subsection{Autres}
			\cvitem{Systèmes d’exploitation}{GNU/Linux, Microsoft Windows}
			\cvitem{Suites logiciels}{Visual Studio, JetBrains, Eclipse, Unity, git, Microsoft Office, Adobe CC, Xcas, Maxima\ldots}

%-------------------------------------------
%   Projects
%-------------------------------------------

  \vspace*{\deletedSpace}
	\section{Projets}
		\cvline{Mini-jeux}{Snake like multijoueur (réseau local) en C++ (SFML), Dungeon crawler en JAVA (JavaFx), Simulateur de gestion de métros en JAVA (JavaFx), Jeu de la vie de Conway en SARL (langage orienté agents).}
		\cvline{Utilitaires}{Serveur de stockage de fichiers multi-utilisateurs en C++ (SFML), Captation des mouvement d'une personne et transmission réseau vers un avatar en réalité augmentée avec Unity (scripting en C\#), Kinect et Vuforia.}

%-------------------------------------------
%   Languages
%-------------------------------------------

  \vspace*{\deletedSpace}
	\section{Langues}
		\cvdoubleitem{\bfseries Français}{\cvskill{}{5} Langue maternelle}{\bfseries Espagnol}{\cvskill{}{3} B1, niveau scolaire}
		\cvdoubleitem{\bfseries Anglais}{\cvskill{}{4} B2, usage professionnel}{\bfseries Japonais}{\cvskill{}{2} A2, élémentaire}

%-------------------------------------------
%   Work experience
%-------------------------------------------

  \vspace*{\deletedSpace}
	\section{Expériences Professionnelles}
		\tllabelcventry{2016/2}{2016/2}{Févr 2016}{Animateur}{JAB France}{Evolène, Suisse}{Camp de ski de 40 jeunes, équipe de 15 animateurs}{}
		\tllabelcventry{2015/2}{2015/3}{Janv--Févr 2015}{Animateur}{JAB France}{Contamines, France}{Camp de ski de 50 jeunes, équipe de 20 animateurs}{}
		\tllabelcventry{2015/1}{2015/2}{Janv--Févr 2015}{Stagiaire}{Souchier SAS}{Héricourt, France}{Jointage et montage d'appareils de désenfumage}{}

%-------------------------------------------
%   Interests
%-------------------------------------------

  \vspace*{\deletedSpace}
	\section{Centres d'intérêts}
		\cvline{L'informatique}{Les nouveaux paradigmes de programmation, la génération procédurale, l'intelligence artificielle, la cryptographie\ldots{}, leur mise en pratique, et l'évolution de l'informatique quantique.}
		\cvline{Les sciences}{Les mathématiques, l'informatique théorique, la physique théorique\ldots}
		\cvline{Le sport}{Ski, vélo, marche\ldots}

\end{document}
